\chapter{Introdução}

%=====================================================

% A introdução geral do documento pode ser apresentada através das seguintes seções: Desafio, Motivação, Proposta, Contribuição e Organização do documento (especificando o que será tratado em cada um dos capítulos). O Capítulo 1 não contém subseções\footnote{Ver o Capítulo \ref{cap-exemplos} para comentários e exemplos de subseções.}.

Classificação em geral sempre foi uma tarefa de descrição simples, porém essencial no dia a dia, podendo ser muito desafiadora. Um dos problemas de
classificação com grandes desafios é o problema da administração de grandes estacionamentos ao ar livre e a necessidade de se determinar se uma vaga de 
estacionamento está ocupada ou livre. Existem algumas formas de resolver esse problema, porém, sem dúvidas a de menor custo e maior facilidade de implementação
é a útilização de métodos de visão computacional e machine-learning para realizar essa classificação de vagas, dado que, a simples instalação de uma câmera
comum supriria a necessidade de coletar as imagens que serão processadas. Porém, neste contexto outro problema surge: Conseguir uma quantidade grande e 
suficiente (milhares ou milhões) de dados rotulados com qualidade para treinar o modelo, com variações de luz, clima, posição e de estacionamentos diferentes,
de forma que os dados sejam o mais genérico possível.

Coletar esses dados para treinamento de um modelo de machine learning é um processo complicado, demorado e de custo elevado, o que limita a quantidade de dados 
disponíveis e as possibilidades de representação e variação que os dados contém. Geralmente essa limitação gera dados muito parecidos, e a divisão entre dados
de treino e teste também acaba sendo impactada, pois torna os dois conjuntos muito correlacionados. Uma alternativa promissora para esse problema é a utilização
de dados sintéticos para o treinamento dos modelos de machine learning. Um dado sintético é uma informação gerada artificialmente e algoriticamente, não dependente 
de eventos do mundo real. Enquanto reunir dados de alta qualidade da forma convencional é difícil, caro e demorado, dados sintéticos trazem a promessa de 
possibilitar a geração rápida, fácil e barata de dados rotulados em qualquer quantidade desejada, com as mais diversas variações e adaptados as necessidades 
expecíficas dos projetos\cite{TechTarget}.

Neste trabalho é proposta a utilização de dados sintético para o treinamento de modelos para classificação de vagas de estacionamento, utilizando a 

%=====================================================
