\chapter{Introdução}

%=====================================================

% A introdução geral do documento pode ser apresentada através das seguintes seções: Desafio, Motivação, Proposta, Contribuição e Organização do documento (especificando o que será tratado em cada um dos capítulos). O Capítulo 1 não contém subseções\footnote{Ver o Capítulo \ref{cap-exemplos} para comentários e exemplos de subseções.}.

Nos últimos anos, o crescimento acelerado das cidades e a crescente frota de veículos têm levantado o problema de se estacionar carros de forma ágil em vias públicas e grandes espaços de estacionamento e destacado a necessidade de uma gestão eficiente desses espaços. A busca por soluções inovadoras que otimizem o seu uso, proporcionando comodidade aos motoristas e reduzindo congestionamentos, tem sido objeto de extensas pesquisas na última década \citep{systematic-review}\citep{parking-lot-monitoring-technologies}. Dentro desse contexto, soluções de monitoramento por imagem utilizando métodos baseados em visão computacional e aprendizado de máquina são comumente escolhidas devido ao seu baixo custo e facilidade de implantação \citep{systematic-review}\citep{pklot1}\citep{cnrpark}\citep{hochuli-2}, comparado a outras técnicas de gerenciamento como as baseadas em sensores.

No monitoramento por imagens, capturas de uma alta e ampla perspectiva servem para monitorar uma grande área de estacionamento. Cada vaga é identificada e segmentada da imagem e modelos de classificação baseados em aprendizado de máquina são utilizados para classificar as vagas segmentadas como ocupadas ou livres.

Uma das maiores limitações dos modelos de classificação baseados em aprendizado de máquina é lidar com a falta de dados \citep{synthetic-pedestrians}. Os modelos de classificação de vagas de estacionamento demandam uma boa quantidade de imagens com qualidade, diversidade e boa representatividade do mundo real para o treinamento. A coleta dessas imagens é uma tarefa complexa e dispendiosa, requer tempo, recursos financeiros e esforço, além de questões legais que tornam a coleta limitada. Nesse contexto, bases de dados como a PkLot \citep{pklot2} e CNRPark-EXT \citep{cnrpark} foram criadas, a fim de disponibilizar imagens para o treinamento e validação de modelos no problema de classificação de vagas de estacionamento. Experimentos com essas e outras bases de dados apresentaram resultados de em média 95\% de acurácia na classificação com modelos treinados e validados em cenários específicos \citep{systematic-review}\citep{hochuli-1} bem como experimentos com validação cruzada entre bases de dados \citep{hochuli-2}, apresentando uma boa capacidade de generalização. No entanto, dado que é um problema em evolução contínua, a necessidade de novas bases de dados para treinamento e validação de modelos ainda é um problema \citep{systematic-review}.

Uma forma de contornar o problema da falta de dados é a utilização de dados sintéticos \citep{synthetic-pedestrians} \citep{domain-random}. Neste trabalho, propõe-se a utilização de imagens sintéticas no treinamento de modelos de classificação de vagas de estacionamento como forma de sobrepujar os desafios existentes no uso de dados reais. Dados sintéticos são dados gerados de forma artificial e algorítmica, assemelhando-se aos dados reais, embora não surjam de observações diretas ou coletas reais. Isso possibilita superar alguns desafios associados à coleta de dados reais, permitindo a criação de conjuntos de dados volumosos, com maior variedade e a um custo reduzido, em um período de tempo menor e com a capacidade de alcançar e até superar os resultados obtidos com dados reais \citep{objectPose}.

Os experimentos conduzidos neste trabalho tem como objetivo definir um modelo base de Deep Learning confiável e de boa precisão e verificar os impactos do uso de imagens sintéticas no treinamento de modelos para classificação de vagas de estacionamento utilizando esse modelo base. Com isso espera-se poder responder 2 perguntas:
\begin{itemize}
    \item Qual o impacto das imagens sintéticas nos modelos estado da arte de classificação de vagas de estacionamento?
    \item É possível superar o estado da arte treinando os modelos somente com imagens sintéticas?
\end{itemize}

Será utilizado o motor gráfico Unity 3D e seu pacote voltado para visão computacional, o Unity-perception \citep{unity-perception}. Será adotada uma Rede Neural Convolucional (CNN) e utilizado o processo de transfer-learning, tendo a MobileNetv3 \citep{MobileNetV3} pré-treinada na ImageNet como modelo base, para o treinamento com as imagens sintéticas geradas, validando sua acurácia por meio de testes com imagens dos conjuntos de dados PkLot e CNRPark-EXT.

%=====================================================
